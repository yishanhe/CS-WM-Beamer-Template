\documentclass{beamer}
%
% Choose how your presentation looks.
%
% For more themes, color themes and font themes, see:
% http://deic.uab.es/~iblanes/beamer_gallery/index_by_theme.html
%
\mode<presentation>
{
  \usetheme{Pittsburgh}      % or try Darmstadt, Madrid, Warsaw, ...
  \usecolortheme[RGB={19,85,0}]{structure} % or try albatross, beaver, crane, ...
  \usefonttheme{default}  % or try serif, structurebold, ...
  \setbeamertemplate{navigation symbols}{}
  \setbeamertemplate{caption}[numbered]
} 

\usepackage[english]{babel}
\usepackage[utf8x]{inputenc}
\usepackage{url}
% 插入图片
\usepackage{graphicx}
\usepackage{amsmath}
\usepackage{amsfonts}
\usepackage{algorithm}
\usepackage[noend]{algpseudocode}
% 指定存储图片的路径(当前目录下的figures 文件夹)
\graphicspath{{figures/}}


% logo
\logo{\includegraphics[height=0.5cm]{wm-cs.jpg}}%\vspace{220pt}}



\title[GSR2015]{Building fast privacy-preserving scheme for applications on smart devices}
\author{Shanhe Yi }
\institute{ \small{Advisor: Dr. Qun Li} \\ Computer Science \\ College of William and Mary}
\date{March 20, 2015}

\begin{document}

\setbeamertemplate{caption}{\raggedright\insertcaption\par}


\begin{frame}
  \titlepage
\end{frame}


% Uncomment these lines for an automatically generated outline.
\begin{frame}{Outline}
  \tableofcontents
\end{frame}


% Uncomment this, if you do not want the table of contents to pop up at
% the beginning of each subsection:
\AtBeginSubsection[]
{
 \begin{frame}<beamer>{Outline}
   \frametitle{Table of Contents}
   \tableofcontents[currentsection,currentsubsection]
 \end{frame}
}

\AtBeginSection[]
{
    \begin{frame}{Outline}
        \tableofcontents[currentsection,currentsubsection]
    \end{frame}
}

% If you wish to uncover everything in a step-wise fashion, uncomment
% the following command:
%\beamerdefaultoverlayspecification{<+->}



% Structuring a talk is a difficult task and the following structure
% may not be suitable. Here are some rules that apply for this
% solution:

% - Exactly two or three sections (other than the summary).
% - At *most* three subsections per section.
% - Talk about 30s to 2min per frame. So there should be between about
%   15 and 30 frames, all told.

% - A conference audience is likely to know very little of what you
%   are going to talk about. So *simplify*!
% - In a 20min talk, getting the main ideas across is hard
%   enough. Leave out details, even if it means being less precise than
%   you think necessary.
% - If you omit details that are vital to the proof/implementation,
%   just say so once. Everybody will be happy with that.


% presentation starts from here


\section{Introduction}

\begin{frame}{Introduction}

\begin{itemize}
  \item ``Imitator Game''
  \item 
\end{itemize}

%\begin{figure}
%  \centering
%  \includegraphics[width = 0.4\textwidth]{TuringColossus2a.jpg}
%  \caption{Awesome figure \footnote{\url{http://turing.colindaylinks.com/turing2.html}}}
%\end{figure}

% \vskip 1cm

% \begin{block}{Examples}
% Some examples of commonly used commands and features are included, to help you get started.
% \end{block}

\end{frame}

%% Background image
% {
% \usebackgroundtemplate{\includegraphics[width=0.5*\paperwidth]{TuringColossus2a.jpg}}%
% \begin{frame}{The Imitation Game}

% \begin{itemize}
% \item 1
% \item 2
% \item 3
% \end{itemize}
% \end{frame}
% }



\begin{frame}{Nowadays}

\begin{itemize}
\item Internet is liking a blackhole, 
\item Snowden
\item Apple Cloud
\end{itemize}

% Commands to include a figure:
%\begin{figure}
%\includegraphics[width=\textwidth]{your-figure's-file-name}
%\caption{\label{fig:your-figure}Caption goes here.}
%\end{figure}

\begin{table}
\centering
\begin{tabular}{l|r}
Item & Quantity \\\hline
Widgets & 42 \\
Gadgets & 13
\end{tabular}
\caption{\label{tab:widgets}An example table.}
\end{table}

\end{frame}

\subsection{Mathematics}

\begin{frame}{Readable Mathematics}

Let $X_1, X_2, \ldots, X_n$ be a sequence of independent and identically distributed random variables with $\text{E}[X_i] = \mu$ and $\text{Var}[X_i] = \sigma^2 < \infty$, and let
$$S_n = \frac{X_1 + X_2 + \cdots + X_n}{n}
      = \frac{1}{n}\sum_{i}^{n} X_i$$
denote their mean. Then as $n$ approaches infinity, the random variables $\sqrt{n}(S_n - \mu)$ converge in distribution to a normal $\mathcal{N}(0, \sigma^2)$.

\end{frame}

\end{document}
